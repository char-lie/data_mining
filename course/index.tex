\documentclass[a4paper,14pt,russian]{extreport}
%\documentclass[tikz,14pt,russian]{standalone}
% Third parameter is needed for TOC displaying (why?)
\usepackage{../common/dsturep} % оформление по ДСТУ 3008-95
\usepackage{import}
\usepackage{standalone}
\usepackage{comment}
\usepackage{bbm}

\usepackage{tikz}
\usepackage{tikz-3dplot}
\usetikzlibrary{calc}
\usetikzlibrary{plotmarks}
\usepackage{pgfplots}

%\usepackage{scrextend}
\usepackage{changepage}
\usepackage{caption}
\usepackage{listings}
%\usepackage[title,titletoc]{appendix}
%\usepackage{appendix}
\usepackage{longtable}
%\usepackage{slashbox}
\usepackage{diagbox}
\usepackage{lscape}
\usepackage{csvsimple}

\newcommand{\indicatorof}[1]{\mathbbm{1}_{#1}}
\newcommand{\indicator}[1]{\mathbbm{1}\!\left( #1 \right)}
\newcommand{\Indicator}[1]{\mathbbm{1}\!\left\{ #1 \right\}}
\newcommand{\probability}[1]{\mathbb{P}\left( #1 \right)}
\newcommand{\Probability}[1]{\mathbb{P}\left\{ #1 \right\}}
\newcommand{\cdfof}[2]{F^{#1}\left(#2\right)}
\newcommand{\cdf}[1]{\cdfof{}{#1}}
\def\Tau{\mathrm{T}}
\newcommand{\meanof}[2]{\operatorname{M}_{#1} #2}
\newcommand{\Meanof}[2]{\meanof{#1}{\left[ #2 \right]}}
\newcommand{\mean}[1]{\meanof{}{#1}}
\newcommand{\Mean}[1]{\Meanof{}{#1}}
\newcommand{\dispersionof}[2]{\operatorname{D}_{#1} #2}
\newcommand{\Dispersionof}[2]{\dispersionof{#1}{\left[ #2 \right]}}
\newcommand{\dispersion}[1]{\dispersionof{}{#1}}
\def \mcond {\;\middle|\;}
\newcommand{\Covergencen}[1]{\xrightarrow[#1\to\infty]{}}
\newcommand{\cov}[1]{\operatorname{cov}\!\left( #1 \right)}
\DeclareMathOperator*{\argmax}{arg\,max}
\DeclareMathOperator*{\argmin}{arg\,min}

\newcommand{\drawHist}[1]{\begin{tikzpicture}[scale=1]
  \begin{axis}[ymin=-5, ymax=5, xmin=5, xmax=30, ytick=\empty,
    xmajorgrids={true},
    ylabel={Кількість точок}, ylabel near ticks,
    xlabel={час, с}]

\draw[dashed,color=gray!50] ({rel axis cs:0,0}|-{axis cs:0,0}) -- ({rel axis cs:1,0}|-{axis cs:0,0});
\addplot table [x, y, col sep=comma] {data/#1.csv};
\end{axis}
\end{tikzpicture}}

\captionsetup[subfigure]{skip=0ex} % global setting for subfigure

\usepackage{stringenc}
\usepackage{pdfescape}

\makeatletter
\renewcommand*{\UTFviii@defined}[1]{%
  \ifx#1\relax
    \begingroup
      % Remove prefix "\u8:"
      \def\x##1:{}%
      % Extract Unicode char from command name
      % (utf8.def does not support surrogates)
      \edef\x{\expandafter\x\string#1}%
      \StringEncodingConvert\x\x{utf8}{utf16be}% convert to UTF-16BE
      % Hexadecimal representation
      \EdefEscapeHex\x\x
      % Enhanced error message
      \PackageError{inputenc}{Unicode\space char\space \string#1\space
                              (U+\x)\MessageBreak
                              not\space set\space up\space
                              for\space use\space with\space LaTeX}\@eha
    \endgroup
  \else\expandafter
    #1%
  \fi
}
\makeatother
\DeclareUnicodeCharacter{00AD}{-}

\def\chapterConclusion{\section*{Висновки до розділу \arabic{chapter}}
\addcontentsline{toc}{section}{Висновки до розділу \arabic{chapter}}}

%\input{../common/minted.inc}   % оформление листингов в minted
%\bibliographystyle{../common/utf8gost705u}
%\bibliographystyle{../common/utf8gost71u}
\bibliographystyle{../common/utf8gost780u}
%\bibliographystyle{plain}

\usepackage[square,numbers,sort&compress]{natbib}
\renewcommand{\bibnumfmt}[1]{#1.\hfill} % нумерация источников в самом списке — через точку
%\renewcommand{\bibsection}

\usepackage{glossaries}
\makeglossaries


\begin{document}

\import{1_title/}{title.tex}

\clearpage
\setcounter{page}{2}

%\pagestyle{empty}
\tableofcontents
\thispagestyle{empty}

\clearpage
\pagestyle{fancy}

\clearpage

\chapter{Закон Ципфа}

\section{Закон Ципфа}
Отношение ранга слова $R$, то есть его номер в списке слов,
отсортированных по частоте в порядке убывания, к частоте слова $f$,
является постоянным
\begin{equation*}
  Z = R \cdot f,
\end{equation*}
где $f$ --- частота слова в тексте, а $Z$ --- коэффициент Ципфа.
Значит,
\begin{equation*}
  f = \frac{Z}{R}.
\end{equation*}

\section{Задание}

Под понятием ``отфильтровать текст'' тут и далее будут подразумеваться
следующие действия:
\begin{enumerate}
  \item
    очистить текст от всех символов кроме букв и пробелов;
  \item
    буквы привести в нижний регистр, между словами оставить по одному пробелу.
  \item
\end{enumerate}

В лабораторной работе нужно
\begin{enumerate}
  \item
    взять текст (желательно на русском языке)
    длиной более нескольких сотен килобайт;
  \item
    отфильтровать текст;
  \item
    составить частотный словарь слов --- каждому слову текста
    сопоставить количество его повторений в тексте;
  \item
    отсортировать частоты в порядке убывания;
  \item
    изобразить полученные значения на графике,
    выбрав логарифмический масштаб для оси ординат и абсцисс;
  \item
    построить степенную линию тренда и убедиться,
    что график похож на прямую линию, за исключением, возможно,
    ``хвостов'' с обеих концов.
\end{enumerate}

\section{Фильтр}
На Perl написан фильтр, который
\begin{enumerate}
  \item делает заглавные буквы строчными;
  \item убирает всё кроме пробелов, символов табуляций, переносов строк и т.п.;
  \item превращает все символы, которе не являются буквами, в пробел, также
    предотвращает появление двух пробелов подряд.
\end{enumerate}

Вход считывается из stdin, выход происходит в stdout.

\lstset{inputencoding=utf8, extendedchars=\true}
\lstinputlisting[language=perl,caption=filter.pl,
                 numbers=left]{../lab1/filter.pl}

\section{Частотный словарь}
На Python написан скрипт, который составляет частотный словарь
и выводит его в формате csv.
Полученный результат можно открыть в программе для работы
с электронными таблицами для построения графиков.

Вход считывается из stdin, выход происходит в stdout.

\lstset{inputencoding=utf8, extendedchars=\true}
\lstinputlisting[language=python,caption=counter.py,
                 numbers=left]{../lab1/counter.py}

\section{График}
\begin{figure}[h]
  \centering
  \includegraphics[width=.75\textwidth]{../lab1/chart}
  \caption{Результат}
\end{figure}

\chapter{Закон Хипса}

\section{Закон Хипса}
Объём словаря уникальных слов $\nu\left( n \right)$
для текста длиной $n$ связан с длиной текста следующим соотношением
\begin{equation*}
  \nu\left( n \right) = \alpha \cdot n^\beta,
\end{equation*}
где $\alpha$ и $\beta$ --- эмпирические константы,
которые разнятся от языка к языку,
и для европейских языков колеблятся в пределах
от $10$ до $100$ и от $0.4$ до $0.6$ соответственно.

\section{Задание}
В лабораторной работе нужно
\begin{enumerate}
  \item
    взять текст (желательно на русском языке)
    длиной более нескольких сотен килобайт;
  \item
    отфильтровать текст;
  \item
    построить зависимость количества уникальных слов в тексте
    от его размера; для этого достаточно использовать один и тот же текст,
    изымать из него всё больше и больше слов с каждой итерацией,
    и подсчитывать число уникальных слов на каждом шаге;
  \item
    изобразить полученные значения на графике;
  \item
    построить степенную линию тренда и убедиться,
    что полученные параметры $\alpha$ и $\beta$ близки к
    теоретическим значениям.
\end{enumerate}

\section{Фильтр}
На Perl написан фильтр, который
\begin{enumerate}
  \item делает заглавные буквы строчными;
  \item убирает всё кроме пробелов, символов табуляций, переносов строк и т.п.;
  \item превращает все символы, которе не являются буквами, в пробел, также
    предотвращает появление двух пробелов подряд.
\end{enumerate}

Вход считывается из stdin, выход происходит в stdout.

\lstset{inputencoding=utf8, extendedchars=\true}
\lstinputlisting[language=perl,caption=filter.pl,
                 numbers=left]{../lab2/filter.pl}

\section{Частотный словарь}
На Python написан скрипт, который считает зависимость между объёмом текста
и объёмом словаря уникальных слов и выводит его в формате csv.
Полученный результат можно открыть в программе для работы
с электронными таблицами для построения графиков.

\lstset{inputencoding=utf8, extendedchars=\true}
\lstinputlisting[language=python,caption=counter.py,
                 numbers=left]{../lab2/count.py}

\section{График}
\begin{figure}[h]
  \centering
  \includegraphics[width=.95\textwidth]{../lab2/chart}
  \caption{Результат}
\end{figure}

\chapter{$TF-IDF$}

\section{$TF-IDF$}
Для $i$ слова ($n$-граммы) индексы $TF$ и $IDF$ считаются
по следующим формулам, где
$D$ --- множество документов,
$n_k$ --- количество повторений $k$ слова ($n$-граммы) в текущем документе
\begin{equation*}
  \begin{split}
    TF_i  &= \frac{n_i}{\sum_{k} n_k}, \\
    IDF_i &= \log {\frac{\left| D \right|}{
                   \left| \left\{ d \mid t_i \in d \in D \right\} \right|}}.
  \end{split}
\end{equation*}
Сам индекс $TF-IDF$ является произведением индексов $TF$ и $IDF$
\begin{equation*}
  TF-IDF_i = TF_i \cdot IDF_i
\end{equation*}

\section{Задание}

\subsection{Основное задание}
В лабораторной работе нужно
\begin{enumerate}
  \item
    взять текст (желательно на русском языке)
    длиной более нескольких сотен килобайт;
  \item
    отфильтровать текст;
  \item
    подсчитать $TF-IDF$ для каждого слова;
  \item
    изобразить полученные результаты в виде таблицы,
    отсортировав по значению $TF-IDF$ в порядке убывания.
\end{enumerate}

То же самое нужно проделать с биграммами и триадами слов.
Например, в тексте ``мама мыла раму'' биграммы следующие:
``мама мыла'' и ``мыла раму''.

\subsection{Стоп-слова (шумовые слова)}
Стоп-слова --- те слова, которые не несут смысловую нагрузку.
К ним относятся предлоги, частицы и прочее,
если анализируемый документ не является учебником русского языка.

Список стоп-слов можно найти в интернете.
Например, в разделе 12.9.4 Full-Text Stopwords документации к MySQL 5.5
находится список англоязычных шумовых слов.

Для увеличения скорости и уменьшения объёма обрабатываемых данных
\begin{enumerate}
  \item
    при подсчёте $TF-IDF$ для слов можно выбросить из рассмотрения те,
    которые находятся в списке стоп-слов;
    например, слово ``не'' имеет мало смысла в сказке о царе Салтане,
    чего не скажешь о слове ``лебедь'';
  \item
    при подсчёте $TF-IDF$ для биграмм следует исключать те биграммы,
    которые содержат в себе шумовые слова;
    например, биграмма ``я пришёл'' имеет мало смысловой нагрузки,
    но биграмма ``пришёл домой'' скажет больше;
  \item
    при подсчёте $TF-IDF$ для триад следует исключать те элементы,
    которые оканчиваются или начинаются на шумовые слова;
    скажем, ``и она решила'' мало о чём говорит,
    триада ``она решила пойти'' скажет больше,
    но ``решила пойти домой'' несёт определённый смысл.
\end{enumerate}

\section{Фильтр}
На Perl написан фильтр, который
\begin{enumerate}
  \item делает заглавные буквы строчными;
  \item убирает всё кроме пробелов, символов табуляций, переносов строк и т.п.;
  \item превращает все символы, которе не являются буквами, в пробел, также
    предотвращает появление двух пробелов подряд.
\end{enumerate}

Вход считывается из stdin, выход происходит в stdout.

\lstset{inputencoding=utf8, extendedchars=\true}
\lstinputlisting[language=perl,caption=filter.pl,
                 numbers=left]{../lab3/filter.pl}

\section{Счётчик $TF-IDF$}
На Python написан скрипт, который считает $TF-IDF$ для слов
и выводит их в формате csv.

Полученный результат можно открыть в программе для работы
с электронными таблицами для сортировки и фильтрации.

\lstset{inputencoding=utf8, extendedchars=\true}
\lstinputlisting[language=python,caption=counter.py,
                 numbers=left]{../lab3/counter.py}

\section{Результат}
На рисунке \ref{fig:tfidf:words:table} изображены первые $40$ строк таблицы
со значениями $TF-IDF$ для слов из $144$ документов автора
Льва Николаевича Толстого, $27$ документов Фёдора Михайловича Достоевского
и $31$ документа Александра Сергеевича Пушкина, отсортированных по значению
$TF-IDF$ в порядке убывания.

Объём документов Толстого $18$MB, Достоевского $7.6$MB, Пушкина --- $2.8$MB.
Фильтрация происходит соответственно $10.3$, $3.2$ и $2$ секунды.
Далее каждый документ имеет только один перенос строки,
который говорит об окончании документа, и их можно объединить в один файл.
Подсчёт $TF-IDF$ происходит за $6.5$ секунд,
на выходе получается $.csv$ файл объёмом $39$MB.

%На рисунке \ref{fig:tfidf:triad:table} изображены первые $40$ строк таблицы
%с триадами.

\begin{figure}[h]
  \centering
  \includegraphics{../lab3/table}
  \caption{Результат для слов}
  \label{fig:tfidf:words:table}
\end{figure}

%\begin{figure}[h]
%  \centering
%  \includegraphics{../lab3/table}
%  \caption{Результат для триад}
%  \label{fig:tfidf:triad:table}
%\end{figure}

%\csvautotabular{lab.csv}
\csvreader[tabular=|l|l|l|c|,
  table head=\hline № & Книга & Слово & $TF-IDF$ \\\hline,
  late after line=\\\hline]
  {lab.csv}{book=\book,word=\word,tfidf=\tfidf}
  {\thecsvrow & \book & \word & \tfidf}

\end{document}

